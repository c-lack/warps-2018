\documentclass[xcolor={dvipsnames,svgnames}]{beamer}
\usetheme{default}
\usepackage{graphicx}
\usecolortheme{crane}
\definecolor{codeyellow}{rgb}{0.8,0.6,0}
\setbeamertemplate{itemize item}{\color{codeyellow}$\blacktriangleright$}
\setbeamertemplate{itemize subitem}{\color{codeyellow}$\blacktriangleright$}

\author{Chris Davis, Robert Gowers, Nada Jankovicova, Cameron Lack, Benjamin Miller}
\title{Dynamics of Eusocial Mole-Rats}
\begin{document}
	\begin{frame}
		\titlepage
	\end{frame}
	\begin{frame}
	\frametitle{What is Eusociality?}
	\begin{itemize}
	\item Species lives in a colony with a division of labour between working and reproduction.
	\item Working includes tasks such as gathering food, protecting the colony, expanding the colony and feeding the young.
	\item The most well-known eusocial species are insects, however there two species of eusocial mole rat.
	\end{itemize}
	
	\end{frame}
	\begin{frame}
	\frametitle{Naked Mole-Rats}
	\begin{itemize}
	\item Naked mole-rats have a single queen in each colony of around 20-300 members.
	\item They live for up to 30 years, with the probability of death seemingly unaffected by age.
	\end{itemize}
	\end{frame}
	\begin{frame}
	\frametitle{The Model}
	\begin{itemize}
	\item Discrete time stochastic model where each time step is a quarter of a year.
	\item Population split into three parts: queens, workers and juveniles.
	\item Queens reproduce, workers gather resources, juveniles mature into workers.
	\item If there are insufficient resources, members of the colony die.
	\item If the queen dies, a worker assumes the role of queen.
	\end{itemize}
	\end{frame}
	
\end{document}